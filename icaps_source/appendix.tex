%File: anonymous-submission-latex-2023.tex
\documentclass[letterpaper]{article} % DO NOT CHANGE THIS
% \usepackage[submission]{aaai23}  % DO NOT CHANGE THIS
\usepackage{aaai23}  % DO NOT CHANGE THIS
\usepackage{times}  % DO NOT CHANGE THIS
\usepackage{helvet}  % DO NOT CHANGE THIS
\usepackage{courier}  % DO NOT CHANGE THIS
\usepackage[hyphens]{url}  % DO NOT CHANGE THIS
\usepackage{graphicx} % DO NOT CHANGE THIS
\urlstyle{rm} % DO NOT CHANGE THIS
\def\UrlFont{\rm}  % DO NOT CHANGE THIS
\usepackage{natbib}  % DO NOT CHANGE THIS AND DO NOT ADD ANY OPTIONS TO IT
\usepackage{caption} % DO NOT CHANGE THIS AND DO NOT ADD ANY OPTIONS TO IT
\frenchspacing  % DO NOT CHANGE THIS
\setlength{\pdfpagewidth}{8.5in} % DO NOT CHANGE THIS
\setlength{\pdfpageheight}{11in} % DO NOT CHANGE THIS
%
% These are recommended to typeset algorithms but not required. See the subsubsection on algorithms. Remove them if you don't have algorithms in your paper.
\usepackage{algorithm}
\usepackage{algorithmic}

%
% These are are recommended to typeset listings but not required. See the subsubsection on listing. Remove this block if you don't have listings in your paper.
\usepackage{newfloat}
\usepackage{listings}
\DeclareCaptionStyle{ruled}{labelfont=normalfont,labelsep=colon,strut=off} % DO NOT CHANGE THIS
\lstset{%
	basicstyle={\footnotesize\ttfamily},% footnotesize acceptable for monospace
	numbers=left,numberstyle=\footnotesize,xleftmargin=2em,% show line numbers, remove this entire line if you don't want the numbers.
	aboveskip=0pt,belowskip=0pt,%
	showstringspaces=false,tabsize=2,breaklines=true}
\floatstyle{ruled}
\newfloat{listing}{tb}{lst}{}
\floatname{listing}{Listing}
%
% Keep the \pdfinfo as shown here. There's no need
% for you to add the /Title and /Author tags.
\pdfinfo{
/TemplateVersion (2023.1)
}

% DISALLOWED PACKAGES
% \usepackage{authblk} -- This package is specifically forbidden
% \usepackage{balance} -- This package is specifically forbidden
% \usepackage{color (if used in text)
% \usepackage{CJK} -- This package is specifically forbidden
% \usepackage{float} -- This package is specifically forbidden
% \usepackage{flushend} -- This package is specifically forbidden
% \usepackage{fontenc} -- This package is specifically forbidden
% \usepackage{fullpage} -- This package is specifically forbidden
% \usepackage{geometry} -- This package is specifically forbidden
% \usepackage{grffile} -- This package is specifically forbidden
% \usepackage{hyperref} -- This package is specifically forbidden
% \usepackage{navigator} -- This package is specifically forbidden
% (or any other package that embeds links such as navigator or hyperref)
% \indentfirst} -- This package is specifically forbidden
% \layout} -- This package is specifically forbidden
% \multicol} -- This package is specifically forbidden
% \nameref} -- This package is specifically forbidden
% \usepackage{savetrees} -- This package is specifically forbidden
% \usepackage{setspace} -- This package is specifically forbidden
% \usepackage{stfloats} -- This package is specifically forbidden
% \usepackage{tabu} -- This package is specifically forbidden
% \usepackage{titlesec} -- This package is specifically forbidden
% \usepackage{tocbibind} -- This package is specifically forbidden
% \usepackage{ulem} -- This package is specifically forbidden
% \usepackage{wrapfig} -- This package is specifically forbidden
% DISALLOWED COMMANDS
% \nocopyright -- Your paper will not be published if you use this command
% \addtolength -- This command may not be used
% \balance -- This command may not be used
% \baselinestretch -- Your paper will not be published if you use this command
% \clearpage -- No page breaks of any kind may be used for the final version of your paper
% \columnsep -- This command may not be used
% \newpage -- No page breaks of any kind may be used for the final version of your paper
% \pagebreak -- No page breaks of any kind may be used for the final version of your paperr
% \pagestyle -- This command may not be used
% \tiny -- This is not an acceptable font size.
% \vspace{- -- No negative value may be used in proximity of a caption, figure, table, section, subsection, subsubsection, or reference
% \vskip{- -- No negative value may be used to alter spacing above or below a caption, figure, table, section, subsection, subsubsection, or reference

\setcounter{secnumdepth}{2} %May be changed to 1 or 2 if section numbers are desired.

% The file aaai23.sty is the style file for AAAI Press
% proceedings, working notes, and technical reports.
%

% Title

% Your title must be in mixed case, not sentence case.
% That means all verbs (including short verbs like be, is, using,and go),
% nouns, adverbs, adjectives should be capitalized, including both words in hyphenated terms, while
% articles, conjunctions, and prepositions are lower case unless they
% directly follow a colon or long dash
\title{Appendix: Planning with Multi-agent Belief using Justified Perspectives}
\author{
    %Authors
    % All authors must be in the same font size and format.
    % Written by AAAI Press Staff\textsuperscript{\rm 1}\thanks{With help from the AAAI Publications Committee.}\\
    % AAAI Style Contributions by Pater Patel Schneider,
    % Sunil Issar,\\
    Guang Hu,
    Tim Miller,
    Nir Lipovetzky
}
\affiliations{
    %Afiliations
    % \textsuperscript{\rm 1}Association for the Advancement of Artificial Intelligence\\
    % If you have multiple authors and multiple affiliations
    % use superscripts in text and roman font to identify them.
    % For example,

    % Sunil Issar, \textsuperscript{\rm 2}
    % J. Scott Penberthy, \textsuperscript{\rm 3}
    % George Ferguson,\textsuperscript{\rm 4}
    % Hans Guesgen, \textsuperscript{\rm 5}.
    % Note that the comma should be placed BEFORE the superscript for optimum readability
    
    School of Computing and Information Systems\
    The University of Melbourne\\
    Parkville, VIC 3010, AUS\\
    % email address must be in roman text type, not monospace or sans serif
    ghu1@student.unimelb.edu.au, tmiller@unimelb.edu.au, nir.lipovetzky@unimelb.edu.au
%
% See more examples next
}


\usepackage{amsthm}
\usepackage{amssymb}
\usepackage{multirow}
\usepackage{multicol}
\usepackage{booktabs}
\usepackage{amsmath}
\usepackage{verbatim}

% \newcommand{\showPro}{Proposition}
% \newtheoremstyle{propositionstyle}% name
%   {5pt}%      Space above
%   {5pt}%      Space below
%   {}%         Body font
%   {}%         Indent amount (empty = no indent, \parindent = para indent)
%   {\itshape\bfseries}% Thm head font
%   {:}%        Punctuation after thm head
%   {\newline}%     Space after thm head: " " = normal interword space;
%         %       \newline = linebreak
%   {}%         Thm head spec (can be left empty, meaning `normal')
% \theoremstyle{propositionstyle}
% \newtheorem{Proposition}{Proposition}[section]
\theoremstyle{definition}
% \newtheorem{proposition}{Proposition}[section]

% \theoremstyle{definition}
\newtheorem{definition}{Definition}[section]
% \theoremstyle{definition}
\newtheorem{example}{Example}[section]
% \theoremstyle{definition}
\newtheorem{theorem}{Theorem}[section]
%\theoremstyle{definition}
\newtheorem{lemma}{Lemma}[section]
\newtheorem{plan}{Plan}[section]

% \usepackage{multicolumn}
\usepackage{listings}

\newcommand{\dom}{\textrm{dom}}

\newcommand{\timestamp}{ts}
\newcommand{\seq}{\Vec{s}}
% \newcommand{\bool}{\mathbb{B}}
\newcommand{\bool}{\{true,false\}}
\newcommand{\sees}{\triangleright}
\newcommand{\f}{\mathit{f}}
\newcommand{\observation}{\mathit{O}}
\newcommand{\memorization}{\mathit{R}}
\newcommand{\seen}{\mathbb{S}}
\newcommand{\unseen}{\neg \mathbb{S}}
\newcommand{\notknown}{\neg \mathbb{K}}
\newcommand{\cc}{\mathit{c}}

\def\L{\mathcal{L}}
\def\T{T}
\def\unknown{\frac{1}{2}}
\def\NF{\mathcal{N}\hspace{-0.5mm}\mathcal{F}}


\usepackage[textsize=scriptsize,textwidth=1cm]{todonotes}
% \usepackage[disable]{todonotes}
\newcommand{\tm}[1]{\vspace{0.5em}\todo[author=Tim,inline,color=green]{#1}\vspace{0.5em}}
\newcommand{\nl}[1]{\vspace{0.5em}\todo[author=Nir,inline,color=blue!20!white]{#1}\vspace{0.5em}}
\newcommand{\gh}[1]{\vspace{0.5em}\todo[author=Guang,inline,color=red!30!white]{#1}\vspace{0.5em}}
\newcommand{\rw}[1]{\vspace{0.5em}\todo[author=Reviewer,inline,color=yellow!30!white]{#1}\vspace{0.5em}}



\usetikzlibrary{calc,fadings,decorations.pathreplacing,automata}

\begin{document}

\maketitle
\begin{table*}[th]
    \addtolength{\tabcolsep}{-3pt}    
    \centering
    \small
    
        \begin{tabular}{lccccrrrrrl}%cc}
        \toprule
        
        \multirow{3}{*}{}
        & \multicolumn{4}{c}{Parameters} & \multicolumn{5}{c}{Performance} &  \\
         \cmidrule(lr){2-5} \cmidrule(lr){6-10}
        & \multirow{2}{*}{$|Agt|$} & \multirow{2}{*}{$d$} & \multirow{2}{*}{$|\mathcal{G}|$} & \multirow{2}{*}{$|P|$} & \multirow{2}{*}{$|Gen|$} & \multirow{2}{*}{$|Exp|$} & \multirow{2}{*}{$|Calls|$} & \multicolumn{2}{c}{TIME(s)} &\multirow{2}{*}{Goal} \\
        & & & & & & & & {$calls$} & Total & \\
        \midrule
        C01 & $2$ & $1$ & $1$ & $1$ & $3$ & $2$ & $2$ & $0.0$ & $0.0$ & $B_a coin=head $ \\
        C02 & $2$ & $1$ & $1$ & $2$ & $7$ & $18$ & $7$ & $0.0$  & $0.0$ & $B_a coin=tail$ \\
        C03 & $2$ & $1$ & $1$ & $2$ & $5$ & $12$ & $7$ & $0.0$ & $0.0$ & $B_a coin=head \land B_b coin=head$ \\
        C04 & $2$ & $2$ & $1$ & $4$ & $43$ & $126$ & $61$ & $0.0$ & $0.0$& $ B_a coin=head \land B_b coin=tail $ \\
        C05 & $2$ & $3$ & $2$ & $4$ & $46$ & $135$ & $58$ & $0.0$ & $0.0$ & $ B_a coin=head \land B_b coin=tail \land B_b\ B_a coin=head $ \\
        C06 & $2$ & $2$ & $2$ & $6$ & $414$ & $1239$ & $533$ & $0.5$ & $0.7$ & $ B_a\ B_b coin=tail \land B_b B_a coin=head$ \\
        
        % C07 & $2$ & $1$ & $1$ & $2$ & $43$ & $126$ & $52$ & $0.0$ & $0.0$& $UB_{a,b} coin=head \land B_b coin=tail$ \\

        \bottomrule
    \end{tabular}
    \caption{Experimental results for coin domain}
    \label{tab:coin}
\end{table*}
\section{Complete semantics for justified belief}

\begin{definition}[Complete semantics]
The complete semantics for justified belief is defined as:

\vspace{2mm}
\noindent
\begin{tabular}{llll}
 (a) & $(M,\seq) \vDash r(\Vec{t})$      & iff & $\pi(s_n, r(\Vec{t})) = true$\\[1mm]
 (b) & $(M,\seq) \vDash \phi \land \psi$ & iff & $(M,\seq) \vDash \phi$ and $(M,s) \vDash \psi$\\[1mm]
 (c) & $(M,\seq) \vDash \neg \varphi$    & iff & $(M,\seq) \not\vDash \varphi$\\[1mm]
 (d) & $(M,\seq) \vDash S_i v$           & iff & $v \in \dom(\observation_i(s_n))$\\[1mm]
 (e) & $(M,\seq) \vDash S_i \varphi$     & iff & $\forall \vec{g} \in \vec{S}_G, (M,  \vec{g}[\langle\observation_i(s_n)\rangle] ) \vDash \varphi$ or \\
     &                                   &     & $\forall \vec{g} \in \vec{S}_G, (M,  \vec{g}[\langle\observation_i(s_n)\rangle] ) \vDash \neg\varphi$\\[1mm]
 (f) & $(M,\seq) \vDash K_i\varphi$      & iff & $(M, \seq ) \vDash \varphi \land S_i \varphi$\\[1mm]
 (g) & $(M,\seq) \vDash B_i \varphi$       & iff & $\forall \vec{g} \in \vec{S}_G, (M,  \vec{g}[\f_i(\seq)]) \vDash \varphi$  \\[1mm]
\end{tabular}
\vspace{2mm}

\noindent where: $\vec{S}_G \in \vec{S}$ is the set of all possible global states sequences and $\vec{g}[\seq] = g_1[s_1],\dots,g_n[s_n]$; and, $g[s]$ means function override: $g[s](v) = s(v)$ when $v \in \dom(s)$ and $g(v)$ otherwise; and $s_n$ is the final state in sequence $\seq$; that is, $s_n = \seq(|\seq|)$.
\end{definition}

\section{Ternary semantics for justified belief}

\begin{definition}[Ternary semantics]
The ternary semantics are defined using function $\T$, omitting the model $M$ for readability:

\vspace{2mm} 
\noindent
\begin{tabular}{l@{~~}l@{~~}l@{~~}l}
 (a) & $\T[\seq, r(\Vec{t})]$ & $=$ & 1 if $\pi(s_n, r(\Vec{t})) = true$;\\
     &                          &     & 0 if $\pi(s_n, r(\Vec{t})) = false$;\\
     &                          &     & $\unknown$ otherwise\\[1mm] 
 (b) & $\T[\seq, \phi \land \psi]$ & $=$ & $\min(\T[\seq, \phi], \T[\seq, \psi])$\\[1mm]
 (c) & $\T[\seq, \neg \varphi]$    & $=$ & $1 - \T[\seq, \varphi]$\\[1mm]
 (d) & $\T[\seq, S_i v]$           & $=$ & $\unknown$ if $ i \notin \dom(s_n)$ or $ v \notin  \dom(s_n)$\\[1mm]
     &                          &   & $0$ if $v \notin \dom(\observation_i(s_n))$\\
     &                          &   & $1$ otherwise     \\
 (e) & $\T[\seq, S_i \varphi]$    
 & $=$ & $\unknown$ if $\T[\seq,\varphi] = \unknown$ or $ i \notin \dom(s_n)$;\\
     &                          &   & $0$ if $\T[\langle\observation_i(s_n)\rangle, \varphi] = \unknown$;\\
     &                          &   & $1$ otherwise\\[1mm]
  (f) & $\T[\seq, K_i \varphi]$ & = & $ \T[\seq, \varphi \land S_i\varphi]$\\[1mm]
  (g) & $\T[\seq, B_i \varphi]$ & = & $ \T[\f_i(\seq), \varphi]$
\end{tabular}
\vspace{2mm} 

\noindent
where $s_n$ is the final state in sequence $\seq$; that is, $s_n = \seq(|\seq|)$.
\end{definition}

\section{Proof for KD45$_n$}

Now, we give the theorem and proof for KD45$_n$ properties.
\begin{theorem}
\label{pos:kd45}
The following axioms hold, making this a KD45$_n$ logic:

\noindent
\vspace{2mm}
\begin{tabular}{l@{~~}l}
	K (Distribution):           & $B_i \varphi \land B_i(\varphi \rightarrow \psi) \rightarrow B_i \psi $\\[1mm]
	D (Consistency):            & $B_i \varphi \rightarrow \neg B_i \neg \varphi $\\[1mm]
	4 (Positive Introspection): & $B_i \varphi \rightarrow  B_i B_i \varphi $\\[1mm]
	5 (Negative Introspection): & $\neg B_i \varphi \rightarrow B_i \neg B_i \varphi $\\[1mm]
\end{tabular} 
\end{theorem}

% \tm{Move the proof to the appendix to give us some space in the main paper}
\begin{proof}
    Based on the definition of $B_i$, $M,\seq \vDash B_i \varphi$ is equivalent to $M,\f_i(\seq) \vDash \varphi$.
    From this, axiom K is: $M,\f_i(\seq) \vDash \varphi$ and $M,\f_i(\seq) \vDash (\varphi \rightarrow \psi)$ imply $M,\f_i(\seq) \vDash \psi$, which holds trivially.
    For the axiom D, when $M,\f_i(\seq) \varphi$ holds, then it must be that  $M,\f_i(\seq) \neg \varphi$ does not hold.

    Axioms 4 and 5 are more involved.
    The value of a variable from $\f_i(\seq)$ depends on two values: $lt$ and $\seq$, which are, respectively, the last time the variable was seen by agent $i$ and the input perspectives.
    The $lt$ depends only on the function $\observation_i$.
    Since $\observation_i(s)=\observation_i(\observation_i(s))$, the $lt$ of $\f_i(\seq)$ and $\f_i(\f_i(\seq))$ for each state and each variable are the same.
    
    Now, note that the retrieval function $\memorization$ returns the value $v=e$ if $v$ is in the state $s_{lt}$ (the first line of $\memorization$), the value of each variable in $\f_i(s)$ is the same as its in $\f_i(\f_i(s))$. Therefore, $\f_i(s) = \f_i(\f_i(s))$. 

    Given that axiom 4 is equivalent to $M,\f_i(\seq) \vDash \varphi$ implies $M,\f_i(\f_i(\seq)) \vDash \varphi$, this holds trivially.
    
    For axiom 5, $M, \f_i(s) \nvDash \varphi$ is equivalent to $M, \f_i(\f_i(s)) \nvDash \varphi$.
    Based on the definition of $B_i$, $M,\f_i(\f_i(s)) \nvDash \varphi$ gives $M,\f_i(s) \nvDash B_i \varphi$.
    Then, based on the definition of $\neg$, we have that $M,\f_i(s) \nvDash B_i \varphi$ is equivalent to $M,\f_i(s) \vDash \neg B_i \varphi$. Therefore, axiom 5 holds.
\end{proof}

\section{Experiment Results for Coin example}



\end{document}


