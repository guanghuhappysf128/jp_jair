
\subsection{Group Semantics}
The group semantics for the seeing relation is the same as \cite{Hu2022-ul}, while for the belief, it is much more complex.
We can not simply extend the idea of using group seeing relation to derive the agent's justified perspectives, as the agent might holds different beliefs compared to the group.

\subsubsection{Language} We extend the language defined in Definition~\ref{def:language} by adding group seeing formulae, group knowledge formulae and group belief formulae as follows:

\begin{definition}
\label{def:g_language}
Let $G$ be a group of agent and $\gamma$ be $v \mid \alpha$ for simplicity:
    \[
        \alpha ::= \alpha \mid ES_G \gamma \mid DS_G \gamma \mid CS_G \gamma,
    \]
    \[
        \alpha ::= \alpha \mid EK_G \alpha \mid DK_G \alpha \mid CK_G \alpha,
    \]
    \[
        \varphi ::= \varphi \mid \alpha \mid EB_G \varphi \mid CB_G \varphi \mid UB_G \varphi %\mid DB_G \varphi
    \]
\end{definition}
\vspace{2mm}

$ES_G \gamma$ represents every agent in group $G$ sees $\gamma$.
$DS_G \gamma$ and $CS_G \gamma$ represents agents in group $G$ distributively and commonly sees $\gamma$ respectively.
$EK_G \alpha$, $DK_G \alpha$ and $CK_G \alpha$ are similar except they represents knowledge relations.
$EB_G \varphi$ represents every agent in group $G$ believe $\varphi$, while $UB_G \varphi$ represents that agents as a group $G$ believes $\varphi$.
$CB_G \varphi$ represents all agents in group shared a common belief of $\varphi$.
We omit the distributed belief as it does not make sense syntactically. 
\gh{Not sure if the above makes sense}


\subsubsection{Non-Na\"ive Semantics}
Now, we can simply gives the group semantic for the seeing relations as:
\vspace{2mm}

\begin{tabular}{llll}
	(g) & $M,\seq \vDash ES_G\gamma$      & iff & $\forall i\in G$, $M,\seq \vDash S_i\alpha$\\[1mm]
	(h) & $M,\seq \vDash DS_G v$        & iff & $v \in \dom(\bigcup_{i\in G}\observation_i(s_n))$\\[1mm]
	(i) & $M,\seq \vDash DS_G \varphi$  & iff & $M,s' \vDash \varphi$ \\ 
	    &                               &     & $M,s' \vDash \neg \varphi$ \\
	    &                               & \multicolumn{2}{l}{ , where $s'=\bigcup_{i\in G}\observation_i(s_n)$} \\[1mm]
	(j) & $M,\seq \vDash CS_G v$        & iff & $v \in \dom(\cc\observation(G,s_n))$\\[1mm]
% 	(k) & $M,\seq \vDash CS_G \varphi$ & iff & for all $g \in \mathcal{S}_G$, $(M,g[s']) \vDash \varphi$ \\
% 	& & & or for all $g \in \mathcal{S}_G$, $(M,g[s']) \vDash \neg\varphi$, where $s'= \cc\f(G,s)$.
	
	(k) & $M,\seq \vDash CS_G \varphi$  & iff & $M,s' \vDash \varphi$ \\ 
        &                               &     & $M,s' \vDash \neg \varphi$ \\
        &                               & \multicolumn{2}{l}{ , where~\footnote{The function for fix point $\cc\observation$ is exactly the same as the one in Definition~\ref{def:fix_point} with different notation. } $s'=\cc\observation_G(s_n)$} \\[1mm]
\end{tabular}
\vspace{2mm}
\gh{Should I handle g[n] in the $DS_G v$ or $DS_G v$  }

Then, we can derive semantics for group knowledge using the same equivalent relation as in Definition~\ref{def:group_equiv}.

The group belief semantic is more tricky.
Instead of simply using group observation to generate a justified perspective, we should merge agent's individual justified perspectives.
For example using Example~\ref{example:coin} with Plan~\ref{plan2} (shown in Figure~\ref{fig:coin_plan_B}): 
if we simply using group observation $\observation_{a,b}$, there is no state holds $ES_{a,b} coin$, which means joint justified perspective between $a$ and $b$ would return $None$ for the value of $coin$.
Another example can be found in Example~\ref{example:coin} with Plan~\ref{plan1} (shown in Figure~\ref{fig:coin_plan_A}):
if we simply using common group observation fix point $\cc\observation(\{a,b\},s)$, the most recent state that $CS_{a,b} coin$ holds is $s_2$, and value of coin is $head$, which results in $B_{a,b} coin=head$ holds.
However, it is obvious that $B_b coin=tail$ holds, which is inconsistent with $B_{a,b} coin=head$.

\begin{definition}
    
    \label{def:common-perspective-function}
    A common perspective function for a group of agents $G$, with respect to agents $i,j \in G$, $\f_{G,i,j}: \vec{S} \rightarrow \vec{S}$, is defined as as follows:
    \[
    \f_{G,i,j}([s_0,\dots,s_n])=[s'_0,\dots,s'_n] \text{, where:}
    \]
\noindent
    where for all $t \in [0, n]$ and all $v \in \dom(s_t)$:
    %
    \[
    \begin{array}{l@{\ }l@{\ }l}    
    s'_t & = & \{v \mapsto \memorization([s_0,\dots,s_t, lt,v) \mid lt = max(a\timestamp)\}\\
    a\timestamp & = & \{j \mid M, s_j \vDash CS_G S_i v \land j \leq t \}\\
                & = & \{j \mid v \in \observation_i(\cc\f(G, s_k)) \land j \leq t\}
    \end{array}
    \]

Recall that $\cc\f$ is the fixed-point function for defining common seeing and common knowledge from \citet{Hu2022-ul}.

This is similar to the perspective function $\f_i$ for an individual agent, but it considered the common perspective for a group. The variable $lt$ gives us the most recent timestep in which the group of agents $G$ collectively saw that agent $i$ saw variable $v$, while $\memorization([s_0,\dots,s_t],lt,v)$ is as for the single-agent case. This will be empty if the group $G$ has never collectively seen agent $i$ observe $v$, as it cannot be common belief otherwise.
\end{definition}

We now define $CB_G$, the common belief of group $G$, as follows:

\vspace{2mm}
\noindent
\begin{tabular}{lll}
  $M, \seq \vDash CB_G \varphi$ & iff & for all $i,j \in G$, $M, \f_{G,i}(f_j(\seq)) \vDash \varphi$
\end{tabular}
\vspace{2mm}

This requires some unpacking. The expression $\f_{G,i}(f_j(\seq))$ captures the commonly-believed perspective from agent $j$'s perspective. The perspective $\f_{G,i}(\seq)$ finds the perspective for when the group $G$ saw agent $i$ observing each variable. This effectively gives each agent in $G$ the reference to when agent $i$ saw each variable. However, each agent in the group then needs to determine what value \emph{they} believe that variable is, because they may not have observed some variables in the same state that another agent did. To do this, for each agent $j$, we pass their perspective $\f_j(\seq)$ into $\f_{G,i}$. The expression $\f_{G,i}(f_j(\seq))$ therefore returns agent $j$'s perspective on the common belief. If all agents $j \in G$ agree that $\varphi$ is true in their common perspective, then $\varphi$ is common belief.

Hence, we propose a new concept called unified belief.
An unified belief over formula $\varphi$ among a group of agent $G$ can be represented by $UB_G$.
An unified perspective function can be derived from perspective function defined in Definition~\ref{def:perspective_function} with joint observation $\cc\observation$.

\begin{definition}
    
    \label{def:unified_perspective_function}
    An unified perspective function for a group of agent $G$, $U\f_G: \vec{S} \rightarrow \vec{S}$, is a recursive function as follows:
    \[
    U\f_G([s_0,\dots,s_t,\dots,s_n])=[s'_0,\dots,s'_t,\dots,s'_n] \text{, where:}
    \]
    \[    
    s'_t = \{v \mapsto \memorization([s_0,\dots,s_t],lt,v) \mid lt = max(a\timestamp), v\in V\} \text{,} 
    \]
    \[    
    \text{where: }a\timestamp = \{j \mid v \in \cc\observation_G(s_j) \land j \leq t \}
    \]
    % \[    
    % \text{where: }a\timestamp = \{j \mid \triangleright_i(s_j)(v) = true \land j \leq t \}
    % \]


% \begin{itemize}
%     \item $\f_i([s_0,\dots,s_t])=[s_0',\dots,s_t']$, where:

%     \item $s_t' = \{v \mapsto \memorization_i([s_0,\dots,s_t],lt,v,t) \mid lt = max(st)$, where $st = \{tt\mid \triangleright_i(s_{tt},v) =1 \}\}$

% \end{itemize}
\end{definition}
\vspace{2mm}

Then, in order to handle common belief, we give definition of the perspective fix point as follows:

\begin{definition}
\label{def:fix_perspective}
    Let $\seq$ be the given perspective and the group of agent as $G$, the fixed perspective $\cc\f_G(\seq) = \seq~'$ can be defined as:
    \[
        \cc\f_G(\seq~') = \cc\f_G(\bigcap_{i \in G} \f_i(\seq~'))
    \]
    
\end{definition}


Then, we give the semantics for group belief as follows:
\vspace{2mm}

\begin{tabular}{llll}
	(l) & $M,\seq \vDash UB_G \varphi$  & iff    & $ M,U\f_G(\seq) \vDash \varphi$\\[1mm]
	(m) & $M,\seq \vDash EB_G \varphi$  & iff    & $ M,\seq~'\vDash \varphi$\\
	    &                               & \multicolumn{2}{l}{, where $\seq~'=\bigcap_{i \in G} \f_i(\seq)$ } \\
	    &                               & $\equiv$ & $\forall i \in G,\ M,\seq \vDash B_i \varphi$ \\ [1mm]
% 	(n) & $M,\seq \vDash DB_G \varphi$  & iff    & $ M,\seq~'\vDash \varphi$\\
% 	    &                               & \multicolumn{2}{l}{, where $W=\bigcup_{i \in G} \f_i(\seq)$ } \\
% 	    &                               & $\equiv$ & $\forall i \in G,\ M,\seq \vDash B_i \varphi$ \\ [1mm]
	(n) & $M,\seq \vDash CB_G \varphi$  & iff    & $ M,\seq~'\vDash \varphi$\\
	    &                               & \multicolumn{2}{l}{, where $\seq~'=\cc\f_G(\seq)$ } \\
	    &                               & $\equiv$ & $M,\seq \vDash EB_G \varphi \land UB_G \varphi $ \\ [1mm]	    
\end{tabular}
\vspace{2mm}


% \vspace{2mm}
% \begin{tabular}{llll}
% 	(l) & $M,\seq \vDash UB_G \varphi$      & iff & $M,\f_G(\seq) \vDash \varphi$\\[1mm]
	

% 	(o) & $M,\seq \vDash CS_G v$        & iff & $v \in \dom(\cc\f(G,s))$\\[1mm]
% % 	(k) & $M,\seq \vDash CS_G \varphi$ & iff & for all $g \in \mathcal{S}_G$, $(M,g[s']) \vDash \varphi$ \\
% % 	& & & or for all $g \in \mathcal{S}_G$, $(M,g[s']) \vDash \neg\varphi$, where $s'= \cc\f(G,s)$.
	
% 	(p) & $M,\seq \vDash CS_G \varphi$  & iff & $M,s' \vDash \varphi$ \\ 
%         &                               &     & $M,s' \vDash \neg \varphi$ \\
%         &                               & \multicolumn{2}{l}{ , where~\footnote{The function for fix point $\cc\observation$ is exactly the same as the one in Definition~\ref{def:fix_point} } $s'=\cc\observation(G,s)$} \\[1mm]
% \end{tabular}
% \vspace{2mm}










% \subsection{Group Semantics}

% Now we give the formal definition for our group semantics, which formed by using group observation functions: $E\observation$, $D\observation$ and $C\observation$.

% % \begin{definition}
% % \label{def:group_observation_function}
% %     An Joint Observation Function for a group of agent $g$, $\observation_g: \mathcal{S} \rightarrow \mathcal{S}$ can be defined as: 
% %     \[
% %     E\observation_g(s) = \{ v \rightarrow s(v) \mid \forall i \in g,\ \triangleright_i(s,v) \}
% %     \]
% %     An distributed Observation Function for a group of agent $g$, $\observation_g: \mathcal{S} \rightarrow \mathcal{S}$ can be defined as: 
% %     \[
% %     D\observation_i(s) = \{ v \rightarrow s(v) \mid \exists i \in g,\ \triangleright_i(s,v) \}
% %     \]    
% %     A common observation Function for a group of agent $g$, $\observation_g: \mathcal{S} \rightarrow \mathcal{S}$ can be defined as: 
% %     \[
% %     C\observation_i(s) = \{ v \rightarrow s(v) \mid \triangleright_i(s,v) \}
% %     \]
% % \end{definition}

% % \gh{Which one is better?}
% \begin{definition}
% \label{def:group_observation_function}
%     An Joint Observation Function for a group of agent $g$, $\observation_g: \mathcal{S} \rightarrow \mathcal{S}$ can be defined as: 
%     \[
%     E\observation_g(s) = \bigcap _{i \in G} \observation_i
%     \]
%     An distributed Observation Function for a group of agent $g$, $\observation_g: \mathcal{S} \rightarrow \mathcal{S}$ can be defined as: 
%     \[
%     D\observation_i(s) = \bigcup _{i \in G} \observation_i
%     \]    
%     A common observation Function for a group of agent $g$, $\observation_g: \mathcal{S} \rightarrow \mathcal{S}$ can be defined as: 
%     \[
%     C\observation_i(s) = \cc\observation(g,s)
%     \]
% \end{definition}
% in which $\cc\f(G,s)$ is the state reached by applying composite function $\bigcap _{i \in G} \f_i$ until it reaches its fixed point. That is, the fixed point $s'$ such that $\cc\f(G,s')=\cc\f(G,\bigcap _{i \in G} \f_i(s'))$.

% calculated as follows:
%     \[    
%     a\timestamp = \{j \mid T[s_j, S_i v] =1 \land j \leq t \}
%     \]




% \vspace{2mm}\begin{tabular}{llll}
% %  (f) & $ M,s(t) \vDash \Phi$ &\\ iff & $\bigwedge_{\varphi \in \Phi} M,s(t) \vDash \varphi$\\[1mm]
% %  (f) & $M,s(t) \vDash K_i v$       &\\ iff & $M,s(t) \vDash S_i v \land \pi(s(t),v)=e \land |e|=1$ \\[1mm]
% %  (f) & $M,s(t) \vDash K_i v$       &\\ iff & $M,s(t) \vDash S_i v$ \\[1mm] 
%  (f) & $M,\seq \vDash K_i\varphi$       &\\ iff & $M,\seq \vDash \varphi \land S_i \varphi$ \\[1mm] 
% %  (f) & $M,s(t) \vDash K_i\varphi$       &\\ iff & $M,s(t) \vDash \varphi \land S_i \varphi$ \\[1mm] 
% %  (g) & $M,s(t) \vDash B_i\varphi$   &\\ iff & $M, \f_i(s(t)) \ \vDash \varphi$ \\[1mm]

% %  (h) & if $ M,s(t) \vDash B_i v$ \\ 
% %  & $ M,s(t) \vDash S_i v \lor M,s(t-1) \vDash B_i v$  \\[1mm]
 
% %  (g) & if $ M,s(t) \vDash B_i r(v_1,\dots,v_n)$ & \\
% %  iff 
% % %  & $\bigwedge_{v \in \{v_1,\dots,v_n\}} M,s(t) \vDash B_i v$\\$ \land$ 
% %  & $  M,s(t) \vDash r(\f_i(s(t))(v_1),\dots,\f_i(s(t))(v_n))$  \\[1mm]
%  (g) & if $ M,\seq \vDash B_i \varphi$ & \\
%  iff 
% %  & $\bigwedge_{v \in \{v_1,\dots,v_n\}} M,s(t) \vDash B_i v$\\$ \land$ 
%  & $  M,\f_i(\seq) \vDash \varphi$  \\[1mm] 
% \end{tabular}
% \vspace{2mm}






% \subsection{Non-na\"ive Semantics}

% To handle the issue with local states, similar, we define the expression $a[s]$ to mean function override: $a[s](v)=s(v)$ when $ \triangleright_i(s)(v)$; and, $a(v)$ otherwise. To be specific, $a[s]$ is a list contains every global state $s'$ such that $s \subseteq s'$. Using this we can extend semantics where local states occurs to handled non-visible variables.  
% \vspace{2mm}
% \begin{tabular}{lll}

%     (d) & \multicolumn{2}{l}{ $ M,\seq \vDash S_i v$ }\\ 
%     & iff & $ \triangleright_i(a[s_n])(v) $\\[1mm]
%     (e*) & \multicolumn{2}{l}{$ M,\seq \vDash S_i \varphi$ }  \\ 
%     & iff & $ M,\observation_i(a[s_n])\ \vDash \varphi$ \\ 
%     & or & $ M,\observation_i(a[s_n])\ \vDash \neg\varphi$\\
%  [1mm]
% \end{tabular}
% \vspace{2mm}
% The effect of evaluating $S_i v$ under $a[s_n]$ means that $\triangleright_i(s')(v)$ is evaluated for every state $s'$ from $a[s_n]$. 


% \vspace{2mm}
% \begin{tabular}{lll}

%  (g*) & \multicolumn{2}{l}{ if $ M,\seq \vDash B_i \varphi$ }\\
%  & iff & $  M,\f_i(a[\seq]) \vDash \varphi$  \\[1mm] 
% \end{tabular}
% \vspace{2mm}

%  \gh{It is up to the model to make sure the inconsistent cases would not occur?}
 
%  \gh{possible inconsistent cases: 1, The R needs to be logical separable, otherwise, you need to specify in the prospective function; 2, indicating the false value}






















% \subsubsection{Na\"ive semantics for belief model 2}

% Given this definition of a model, we first define the following semantics for our language, which we call the \emph{na\"{i}ve} semantics, where $s$ is a local or global state:

% \vspace{2mm}

% \begin{tabular}{llll}
%  (a) & $ M,s(t) \vDash r(t_1,\dots,t_k)$ &\\ iff & $\pi(s,t, r(t_1,\ldots,t_k)) = true$\\[1mm]
%  (b) & $ M,s(t) \vDash \phi \land \psi$  &\\ iff & $ M,s(t) \vDash \phi$ and $ M,s(t) \vDash \psi$\\[1mm]
%  (c) & $ M,s(t) \vDash \neg \varphi$     &\\ iff & $ M,s(t) \not\vDash \varphi$\\[1mm]
%  (d) & $ M,s(t) \vDash S_i v$            &\\ iff & $v \in \f_i(s(t))$\\[1mm]
%  (e) & $ M,s(t) \vDash S_i \varphi$     &\\ iff & $ M,\f_i(s(t))\ \vDash \varphi$ or $ M,\observation_i(s(t))\ \vDash \neg\varphi$\\
%  [1mm]
%  (f) & $M,s(t) \vDash K_i\varphi$       &\\ iff & $M,s(t) \vDash \varphi \land S_i \varphi$ \\[1mm] 
%  (g) & $M,s(t) \vDash B_i\varphi$   &\\ iff & $M, s(t) \ \vDash \varphi$ \\[1mm]
% \end{tabular}
% \vspace{2mm}




% \subsection{Ternary Semantics}
% \vspace{2mm} 
% \begin{tabular}{llll}
% (a) & $\T[s(t), r(t_1,\dots,t_k)]$ & $=$ & 1 if $\pi(s,t, r(t_1,\ldots,t_k)) = true$;\\
%  &                          &     & 0 if $\pi(s,t, r(t_1,\ldots,t_k)) = false$;\\
%  &                          &     & $\unknown$ otherwise\\[1mm] % ($vars(r(t_1,\ldots,t_k)) \not\subseteq dom(s)$)
% (b) & $\T[s,t, \phi \land \psi]$  & $=$ & $\min(\T[s,t, \phi], \T[s,t, \psi])$\\[1mm]
% (c) & $\T[s,t, \neg \varphi]$     & $=$ & $1 - \T[s,t, \varphi]$\\[1mm]
% (d) & $\T[s,t, S_i v]$            & $=$ & $1$ if $v \in \dom(f_i(s,t))$; 0 otherwise\\[1mm]
% (e) & $\T[s,t, S_i \varphi]$      & $=$ & $\unknown$ if $\T[s,t,\varphi] = \T[s,t,\neg\varphi] = \unknown$;\\
%  &                          &   & $0$ if $\T[f_i(s,t),t, \varphi] = \T[f_i(s,t),t, \neg\varphi] = \unknown$;\\
%  &                          &   & $1$ otherwise\\ %\max(\T[f_i(s), \varphi], \T[f_i(s), \neg\varphi])$ otherwise\\
% (f) & $\T[s,t, K_i \varphi]$ & $=$ & $\min(\T[s,t, S_i \varphi],\T[s,t, \varphi])$ \\
% (e) & $\T[s,t, B_i \varphi]$      & $=$ & $\T[s,t, K_i \varphi]$ if $\T[s,t,S_i \varphi] = 1\lor t=0$;\\
% %  &                          &   & $\T[s,0, \varphi]$ if $t=0$;\\
%  &                          &   & $\T[s,t-1, B_i \varphi]$ otherwise\\ %\max(\T[f_i(s), \varphi], \T[f_i(s), \neg\varphi])$ otherwise\\
% \end{tabular}
% \begin{lemma}

% \end{lemma}


\vspace{2mm} 

% \section{Forwards method}

% \begin{definition}
%     $Agt = Agt \cup \{g\}$. We denote the omniscience agent as $g$, whose perspective/observation is always consistent with the global states. 
% \end{definition}

% \begin{definition}
%     stats: \{$\seen$,$\unseen$\}
% \end{definition}

% \begin{definition}
%     $\mathcal{S}: V \mapsto D \cup \{\notknown \}, stats$
% \end{definition}



% \begin{tabular}{ll}
%  (1) & $\observation_i(s) \subseteq s$ \\[1mm]
%  (2) & $\observation_i(s) = \observation_i(\observation_i(s))$\\[1mm]
%  (3) & If $s \subseteq s'$, then $\observation_i(s) \subseteq \observation_i(s')$ \\[1mm]
%  (4) & $s = \observation_g(s)$
% \end{tabular}
% \vspace{2mm}



% \begin{definition}
%     $\diamond_{i_s} :  \mathcal{S} \mapsto \mathcal{S}$ is the seen update function, which works as follows: \vspace{2mm}
    
%     $ s \diamond_{i_s} s'$ = $ \{ v \mapsto (e,stat') \mid v \mapsto (e,stat) \in \f_i(s) \}$, 
%      where $stat'$ is $\seen$ if $ \triangleright_i(s',v)$; $stat$ otherwise.
% \end{definition}


% \begin{definition}
%     $\diamond_{i_v} :  \mathcal{S} \mapsto \mathcal{S}$ is the value update function, which works as follows: \vspace{2mm}
    
%     $ s \diamond_{i_v} s'$ = $ \{ v \mapsto (e',stat') \mid v \mapsto (e,stat) \in s \}$, 
%      where:
%      \begin{itemize}
%          \item if $stat = \seen$ and $v \in \observation_i(\f_i(s'))$, then $e' = s'(v), stat' = \unseen$
%          \item otherwise, $e' = e$, $stats' = stats$
%      \end{itemize}
% \end{definition}


% \begin{definition}
%     $s \diamond_i s' = s \diamond_{i_s} s' \diamond_{i_v} s'$
% \end{definition}

% Then, we can define our perspective function as follows:
% \begin{definition}
%     $\f_i : \mathcal{S} \mapsto \mathcal{S}$ is the perspective function returns agents belief of the world based on the sequence of states.
    
%     $\f_i(s_0,\dots,s_t)$: \vspace{2mm}
    
%     \begin{tabular}{ll}
%      (a) & if $t=0$, [$\{ v \mapsto (\notknown,\unseen) \} \diamond_i s_0$ ] \\[1mm]
%      (b) & otherwise, $states$ + $[states[-1]\diamond_i s_t]$-,\\
%          & where $ states = \f_i(s_0,\dots,s_{t-1})$
%     \end{tabular}
    
% \end{definition}


% \begin{itemize}
%     \item [$s_0$:] $\{peeking_a \mapsto f,\ peeking_b \mapsto f,\ coin \mapsto head\}$
%     \item [$\f_b(s_0)$:] $\{peeking_a \mapsto f,\ peeking_b \mapsto f,\ coin \mapsto unknown\}$
%     \item [$\f_b(s_0)$:] $\{peeking_a \mapsto f,\ peeking_b \mapsto f,\ coin \mapsto unknown\}$
% \end{itemize}


% \gh{it should be a sequence of state, not sure about the notation here.}
% \gh{This works exactly the same as sudo code in 3.2. The only problem the complexity.}