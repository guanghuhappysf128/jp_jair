Epistemic planning plays an important role in multi-agent and human-agent interaction domains. 
Most existing works solve multi-agent epistemic planning problems by either pre-compiling them into classical planning problems; or, using explicit actions and their effects to encode Kripke-based semantics. 
A recent approach called \emph{Planning with Perspectives} (PWP) delegates epistemic reasoning in planning to external functions using F-STRIPS, keeping the search within the planning algorithm and lazily evaluating epistemic formulae.
Although PWP is expressive and efficient, it models S5 epistemic logic and does not support belief, including false belief. 
In this paper, we extend the PWP model to handle multi-agent belief by following the intuition that agents believe something they have seen until they see otherwise. We call this \emph{justified perspectives}. We formalise this notion of multi-agent belief based on the definition of knowledge in PWP. Using experiments on existing epistemic and doxastic planning benchmarks, we show that our belief planner can solve benchmarks more efficiently than the state-of-the-art baseline, and can model some problems that are infeasible to model using propositional-based approaches.